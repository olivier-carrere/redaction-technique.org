% To build a PDF version :
% $ xelatex leaflet-piwigo-fr.tex
\documentclass[12pt,nofoldmark,notumble]{leaflet}
\usepackage[utf8]{inputenc}
\usepackage[T1]{fontenc}
\usepackage[francais]{babel}
\usepackage{libertine}
\renewcommand{\familydefault}{\sfdefault}
\usepackage{microtype}
\usepackage{menukeys}
\usepackage{graphicx}
\usepackage{fontawesome}
\usepackage{courier}
\usepackage{titlesec}
\usepackage{color}
\usepackage{verbatim}
\usepackage{enumitem}
\setitemize{label=--}
\setenumerate[1]{label=\textcircled{\scriptsize\arabic*},
  font=\sffamily}
\pagenumbering{gobble}
\CutLine*{3}
\titlespacing\section{0pt}{12pt plus 4pt minus 2pt}{0pt plus 2pt minus 2pt}

% Colors
\definecolor{orange}{RGB}{255,119,0}

\titleformat{\section}
{\color{orange}\normalfont\normalsize\bfseries}
{\color{orange}\thesection}{1em}{}

\begin{document}
\title{Communication}
\date{}
\author{\textsc{Galerie photo}}

\maketitle

\begin{center}
  \includegraphics[height=3cm,keepaspectratio]{lion}%
  \section{Améliorons l'image de l'association !}
\end{center}

\begin{tabular}{lp{7.3cm}}
   \faCloud & Les photos pour la communication de l'association sont centralisées dans le
\emph{cloud} sous une galerie privée \emph{Piwigo}.
\end{tabular}

\begin{tabular}{lp{7.3cm}}

\faPlug &   \begin{enumerate}[itemsep=0mm,leftmargin=*]

  \item Connectez-vous à \keys{\faGlobe  example.com}.

  \item Créez votre compte personnel en cliquant sur \textbf{S'enregistrer}.

  \item Attendez 48 h d'obtenir les droits (sans mail de
    notification), puis connectez-vous dans la zone \emph{Connexion rapide}.

\end{enumerate}
\end{tabular}

\begin{center}
  \includegraphics[height=1cm,keepaspectratio]{Piwigo-logo-black-letters}%
\end{center}

\begin{description}[align=right,labelwidth=2.3cm]
\item [Photographes] \emph{Téléversez} vos meilleures photos !
\item [Iconographes] Cultivez le jardin photo !
\item [Graphistes] Trouvez \textbf{\textit{ze}} photo en 3 clics !
\end{description}

\clearpage
\section{\faCamera  Photographes}

\vspace*{\fill}

\begin{enumerate}[itemsep=0mm,leftmargin=*]

\item Vérifiez que vos photos ne sont pas déjà dans la galerie.
\item Cliquez sur \textbf{Ajoutez des photos}.
  \end{enumerate}
\begin{center}
\setlength{\fboxsep}{0pt}%
\setlength{\fboxrule}{0pt}%
\fbox{\includegraphics[angle=5,width=\linewidth]{photographes}}%
\end{center}
  \begin{enumerate}
  \setcounter{enumi}{2}
\item Ajoutez vos photos et démarrez le transfert.

  Vos photos sont dans l'album \emph{Community}. Elles seront classées par un iconographe.
  
\end{enumerate}

  \vspace*{\fill}

  \fcolorbox{black}{white}{
  \begin{minipage}[t]{1.0\textwidth}

    \section{\faEye  Soyez sélectifs}

      Nous avons déjà plus de 6 000 photos, dont beaucoup se ressemblent (les
      fêtes, ça fait de chouettes photos, mais difficilement utilisables sur
      une affiche\ldots). Privilégiez donc :

      \begin{itemize}
      \item les sujets ou les traitements originaux ;
      \item les gros plans ;
      \item le temps couvert ;
      \item le flou d'arrière-plan ;
      \item les fichiers de plus de 2 Mo ;
      \item les  scans d'argentique ou d'illustrations ;
      \item les boîtiers \emph{reflex}.
      \end{itemize}
      
  \end{minipage}
    }
    \vspace*{\fill}
\clearpage

\section{\faTag  Iconographes}

\vspace*{\fill}

\begin{enumerate}[itemsep=0mm,leftmargin=*]

\item Contactez-nous pour rejoindre l'équipe d'iconographes.
\item Affichez une photo de l'album \emph{Community}.
\item Cliquez sur \faPencil  \textbf{Mots-clés}.
\item Ajoutez des mots-clés aux photos :

  \begin{itemize}
  \item Indiquez \emph{Print} si la photo convient à l'impression, \emph{Web}
    dans le cas contraire.

  \item Pour que la photo soit supprimée, indiquez \emph{Delete}.  Elle sera
    effacée plus tard\footnote{Elle sera conservée dans la sauvegarde.}.
  \end{itemize}
\end{enumerate}
  
\begin{center}
  \setlength{\fboxsep}{0pt}%
  \setlength{\fboxrule}{0pt}%
  \fbox{\includegraphics[angle=5,width=\linewidth]{iconographes}}%
\end{center}

\vspace*{\fill}

\begin{itemize}
  \item[]
  \begin{itemize}
  \item Associez au moins les mots-clés :

    \begin{itemize}
    \item \emph{Photo} ou \emph{Illustration} ;
    \item \emph{Couleur} ou \emph{Monochrome} ;
    \item \emph{Vertical}, \emph{Carré} ou \emph{Horizontal} ;
    \item \emph{Intérieur} ou \emph{Extérieur} ;
    \item \emph{Gros plan} ou \emph{Plan large} ;
    \item \emph{Portrait}, \emph{Groupe}, \emph{Paysage} ou \emph{Nature morte}.
    \end{itemize}
  \end{itemize}
\end{itemize}
\vspace*{\fill}
\clearpage

\section{\faPaintBrush  Graphistes}

\vspace*{\fill}

\begin{enumerate}[itemsep=0mm,leftmargin=*]

\item Cliquez sur \textbf{Mots-clés}.
\item Sélectionnez un mot-clé.
\end{enumerate}
\begin{center}
  \setlength{\fboxsep}{0pt}%
  \setlength{\fboxrule}{0pt}%
  \fbox{\includegraphics[angle=5,width=\linewidth]{graphistes}}%
\end{center}

\begin{enumerate}
  \setcounter{enumi}{2}
\item Affinez votre recherche en sélectionnant d'autres mots-clés.
\item Affichez \textbf{\textit{ze}} photo, puis téléchargez-la \faDownload.
\end{enumerate}

\vspace*{\fill}

\fcolorbox{black}{white}{
  \begin{minipage}[t]{1.0\textwidth}
\section{\faLeaf  Cultiver le jardin photo}

      \emph{Le « désherbage » régulier d'un fonds est toujours
        souhaitable. Comme pour un jardin, il faut veiller à ce que la base de données
        soit toujours en bon état afin d'être efficace.}

      \emph{Il ne suffit pas d'alimenter une base ; il faut s'assurer que les
        résultats qu'elle fournit sont le plus pertinents possible.  De
        nouveaux mots-clés peuvent apparaître, et il peut être plus que
        pertinent de les intégrer. Il est inutile de tout garder ; il faut
        savoir trier.}

      \textsc{Profession iconographe}

      \textsc{Aurélie Lacouchie - Éditions Eyrolles}
      
  \end{minipage}
}
\vspace*{\fill}
\clearpage

\section{\faUniversity  Atelier de création}

Contactez-nous pour participer à la création d'affiches en ligne
avec l'équipe \keys{\faGlobe  canva.com}.

\begin{center}
  \setlength{\fboxsep}{0pt}%
  \setlength{\fboxrule}{0pt}%
  \fbox{\includegraphics[angle=5,width=0.864\linewidth]{canva}}%
\end{center}
\vspace*{\fill}
\section{\faSave  Sauvegarde}

Vous souhaitez protéger le patrimoine photo de l'association ?

Contactez-nous pour cloner le dépôt \emph{Git} de la galerie !
\vspace*{\fill}
\section{\faYoutube  Vidéos explicatives}

\keys{\faGlobe  example.com}

\begin{center}
  \setlength{\fboxsep}{0pt}%
  \setlength{\fboxrule}{0pt}%
  \fbox{\includegraphics[angle=5,width=0.864\linewidth]{didacticiels}}%
 \end{center} 

\clearpage

\begin{center}
  \setlength{\fboxsep}{0pt}%
  \setlength{\fboxrule}{0pt}%
  \fbox{\includegraphics[angle=5,width=\linewidth]{workflow}}%

  Workflow photo
\end{center}

\vspace*{\fill}

\section{\faInfoCircle  Ressources}

\begin{description}[align=right,labelwidth=4.2cm]

\item [Retouche photo] \keys{\faGlobe  gimp.org/fr}

\item [Développement RAW] \keys{\faGlobe  darktable.fr}

\item [PAO] \keys{\faGlobe  scribus.fr}

\item [Logiciels] \keys{\faGlobe  huit.re/freesoft}

\item [Photo] \keys{\faGlobe  tontonphoto.fr}

\end{description}
   
\vspace*{\fill}

\section{\faEnvelope  Contact}

\texttt{adress@example.com}
\vspace*{\fill}
\begin{center}
{ \textsc{Association} \\ 34,
  rue - Code postal \textsc{Ville} - Pays}

\keys{\faGlobe  example.com}
\end{center}

\end{document}

